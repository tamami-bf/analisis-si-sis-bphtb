%%%%%%%%%%%%%%%%%%%% book.tex %%%%%%%%%%%%%%%%%%%%%%%%%%%%%
%
% sample root file for the chapters of your "monograph"
%
% Use this file as a template for your own input.
%
%%%%%%%%%%%%%%%% Springer-Verlag %%%%%%%%%%%%%%%%%%%%%%%%%%


% RECOMMENDED %%%%%%%%%%%%%%%%%%%%%%%%%%%%%%%%%%%%%%%%%%%%%%%%%%%
\documentclass[pdftex,12pt, oneside]{article}

% choose options for [] as required from the list
% in the Reference Guide, Sect. 2.2
%\usepackage[paperwidth=8.5in, paperheight=13in]{geometry} % Folio
\usepackage[paperwidth=8.27in, paperheight=11.69in]{geometry} % A4

\usepackage{makeidx}         % allows index generation
\usepackage{graphicx}        % standard LaTeX graphics tool
                             % when including figure files
%\usepackage{multicol}        % used for the two-column index
\usepackage[bottom]{footmisc}% places footnotes at page bottom
\usepackage[english]{babel}
\usepackage{enumerate}
\usepackage{paralist}
\usepackage{float}
\usepackage{gensymb}  
\usepackage{listings}
%\usepackage{siunitx}
% etc.
% see the list of further useful packages
% in the Reference Guide, Sects. 2.3, 3.1-3.3
\renewcommand{\baselinestretch}{1.5}

\newcommand{\HRule}{\rule{\linewidth}{0.5mm}}

%\makeindex             % used for the subject index
                       % please use the style svind.ist with
                       % your makeindex program


%%%%%%%%%%%%%%%%%%%%%%%%%%%%%%%%%%%%%%%%%%%%%%%%%%%%%%%%%%%%%%%%%%%%%

\begin{document}
\sloppy
%\author{Priyanto Tamami}
%\title{BUKU PETUNJUK OPERASIONAL SISTEM INFORMASI GEOGRAFIS UNTUK PBB-P2 DENGAN MAPINFO VERSI 8.0}
%\date{22 Desember 2015}
%\maketitle

%\input{./01.title.tex}
\begin{center}
{\large ANALISIS SISTEM INFORMASI BPHTB}
\\[1cm]
21 Maret 2016\\
Priyanto Tamami, S.Kom.
\end{center}

%\frontmatter%%%%%%%%%%%%%%%%%%%%%%%%%%%%%%%%%%%%%%%%%%%%%%%%%%%%%%

%\include{dedic}
%\include{pref}

%\include{02.pengesahan} 

%\tableofcontents
%\listoffigures

%\mainmatter%%%%%%%%%%%%%%%%%%%%%%%%%%%%%%%%%%%%%%%%%%%%%%%%%%%%%%%
%\include{part}
%\include{chapter}
%\include{chapter}
%\appendix
%\include{appendix}

%\include{03.konsep-sig}
%\include{04.pengenalan-software}
%\include{05.koordinat}
%\include{06.registrasi-transformasi-koordinat} 
%\include{07.digitasi-on-screen} 
%\include{08.query} 

%\backmatter%%%%%%%%%%%%%%%%%%%%%%%%%%%%%%%%%%%%%%%%%%%%%%%%%%%%%%%
%\include{solutions}
%\include{referenc}
%\printindex

%%%%%%%%%%%%%%%%%%%%%%%%%%%%%%%%%%%%%%%%%%%%%%%%%%%%%%%%%%%%%%%%%%%%%%

\section{SASARAN DAN BATASAN SISTEM}

Sasaran dan batasan dari sistem yang akan dibangun adalah yang pertama tentu saja menggantikan pola pencatatan manual yang masih menggunakan Microsoft Office Excel, dengan aplikasi data \textit{entry} yang terintegrasi dengan basis data Sistem Informasi Manajemen Objek Pajak (SISMIOP) yang dilakukan untuk mengelola data objek dan subjek Pajak Bumi dan Bangunan Perdesaan dan Perkotaan (PBB-P2), sehingga nantinya pada saat \textit{entry} data dapat diketahui apakah Nomor Objek Pajak (NOP) yang akan dilakukan verifikasi Bea Perolehan Hak atas Tanah dan Bangunan (BPHTB) sudah melunasi PBB-P2 terhutang atau belum.

Sistem ini pun nantinya akan menampilkan informasi statistik jenis pelayanan BPHTB yang telah diterima, dan berapa jumlah berkas pengajuan pelayanan BPHTB yang sudah selesai.

Sistem pun akan dibuka untuk akses bagi wajib pajak atau kuasanya, namun hanya untuk pemberitahuan status berkas apakah sudah selesai, atau tertunda karena beberapa hal seperti perlunya pemeriksaan lapangan atau adanya kekurangan kelengkapan berkas yang harus diserahkan.


\section{ARSITEKTUR SISTEM}

Seperti membangun sebuah gedung atau bangunan, agar berdiri kokoh dan memudahkan dalam pemeliharaan maka diperlukan sebuah arsitektur, sedangkan pada pembuatan sebuah sistem aplikasi maka diperlukan sebuah arsitektur sistem aplikasi yang tujuannya pun agar kokoh dalam artian stabil dengan sedikit permasalahan yang muncul, dan memberikan kemudahan pada saat pemeliharaan atau penambahan fasilitas pada sistem aplikasi yang sudah jadi.

Arsitektur sistem akan terbagi menjadi 3 (tiga) bagian, yaitu :

\begin{enumerate}[a.] 
\item Bagian Basis Data

\item Bagian Logika Aplikasi

\item Bagian Tampilan Aplikasi
\end{enumerate}

\section{DESKRIPSI SUB SISTEM}

\subsection{Bagian Basis Data}

Pada bagian ini akan terdapat beberapa tabel baik dari sisi aplikasi SISMIOP maupun dari aplikasi BPHTB yang akan dibangun nantinya, beberapa tabel yang digunakan untuk menyimpan data-data SISMIOP hanya digunakan sebagai bahan pendukung pada data-data transaksi BPHTB. Sedangkan simpanan utama untuk sistem BPHTB ini nantinya akan dibangun dari awal menggunakan basis data Postgresql.

\subsection{Bagian Logika Aplikasi}

Pada bagian ini nantinya akan terdiri dari diagram struktur dan diagram perilaku dari aplikasi, bagaimana masing-masing komponen saling bertukar informasi, bagaimana alur data dari basis data ke bagian tampilan dan sebaliknya dimodelkan pada bagian ini.

Bagian ini bisa disebut inti dari aplikasi, karena bagian ini yang nantinya mengontrol transfer data antar komponen dan lapisan, bagian ini pun yang nantinya memanggil laporan-laporan untuk ditampilkan kepada pengguna.

\subsection{Bagian Tampilan Aplikasi}

Istilah aslinya biasa disebut dengan \textit{User Interface}, bagian ini yang nantinya akan menjadi desain atau model dari tampilan yang berhadapan langsung dengan pengguna. Bagian ini yang nantinya mengumpulkan informasi untuk disampaikan kepada bagian logika aplikasi untuk diproses, bagian ini pula yang nantinya menampilkan informasi yang diproses oleh sistem untuk dapat dibaca dan dipahami oleh pengguna.

\section{PERTIMBANGAN KHUSUS KINERJA SISTEM}

Karena sistem datanya terpusat, yaitu tersimpan pada server basis data Postgresql, maka perawatan atau pemeliharaan mutlak dilakukan pada sisi server basis data yang diakibatkan dari peningkatan jumlah data yang selalu bertambah dari sisi ukuran dari hari ke hari.

Biasanya akan dilakukan pembersihan terhadap \textit{file log}, yaitu \textit{file} yang digunakan oleh sistem basis data postgresql untuk mencatatkan kegiatan atau aktivitas yang telah dilakukannya selama beroperasi melayani permintaan data. Pembersihan ini dilakukan secara rutin karena transaksi yang terjadi pun rutin dilakukan harian.

Selain server basis data, server aplikasi web pun perlu mendapatkan perhatian pada sisi perawatan atau pemeliharaan, dimana server aplikasi web pun memiliki \textit{file log} tersendiri yang perlu dibersihkan secara rutin karena menyimpan catatan transaksi antara komputer \textit{client} dan server.

Hal lain yang perlu dilakukan sebagai langkah pengamanan data yaitu dilakukannya duplikasi data (\textit{backup data}) sehingga apabila terjadi hal-hal yang tidak diinginkan, data dapat langsung dikembalikan atau dipulihkan (\textit{recovery}). Ini pun sebaiknya dilakukan secara rutin, sebaiknya dalam periode harian.

\section{HASIL PEMODELAN}

Hasil pemodelan untuk sistem informasi BPHTB ini nantinya akan dibagi menjadi beberapa diagram mendasar pada beberapa bagian dari arsitektur sistem, berikut rincian mengenai hasil pemodelan tersebut :

\subsection{Bagian Basis Data}

\subsection{Bagian Logika Aplikasi}

Ada beberapa pemodelan logika aplikasi di dunia teknologi informasi, jika pengembangan dilakukan dengan bahasa pemrograman terstruktur, maka akan menggunakan \textit{flowchart} sebagai alat untuk pemodelan, karena yang akan digunakan sekarang adalah bahasa pemrograman Java yang menggunakan metodologi orientasi objek, maka digunakan Unified Modeling Language (UML) yang akan menggambarkan struktur sistem dari awal sampai akhir. Berikut adalah pemodelan atau diagram yang menggambarkan bagaimana sistem informasi BPHTB bekerja.

\subsubsection{Diagram \textit{Use Case}}



\subsubsection{Diagram \textit{Information Flow}}

% liat disini -> http://www.uml-diagrams.org/information-flow-diagrams/scheduled-workflow-example.html

\subsubsection{Diagram \textit{Class}}

\subsubsection{Diagram \textit{Package}}

\subsubsection{Diagram \textit{Component}}

\subsubsection{Diagram \textit{Deployment}}

\subsubsection{Diagram \textit{Communication}}

% liat disini -> http://www.uml-diagrams.org/communication-diagrams-examples.html

\subsubsection{Diagram \textit{Activity}}

\subsubsection{Diagram \textit{Sequence}}

\subsection{Bagian Tampilan Aplikasi}

\section{BIAYA DAN JADWAL PENGEMBANGAN}


\end{document}